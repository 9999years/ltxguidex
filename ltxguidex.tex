\documentclass{ltxguidex}
\usepackage{bera}
\newcommand{\ltx}{\ltxclass{ltxguidex}}
\title{Towards Better \LaTeX\ Documentation With the \ltx\ Document Class}
\author{Rebecca Turner\thanks{Brandeis University;
	\email{rebeccaturner@brandeis.edu}}}
\date{2019-01-07}
\begin{document}
\maketitle

\begin{abstract}
	The \ltx\ document class provides\dots

	This document is written with the \ltx\ document class.
\end{abstract}

\section{The State of the Docs}

\LaTeX\ documentation is easy enough to write that --- in general --- nobody
has bothered to package the ways of making it better. If one examines the
documentation for their favorite package, they'll likely find a few command
definitions that make some aspect of documentation writing more ergonomic.

\LaTeX\ documentation is enabled with two document classes and several
packages. Document classes include:
\begin{classes}
	\item[ltxdoc] Defines very little other than a few
	shorthands for documenting commands. Designed to be integrated with
	the \docstrip\ system, but I've seen plenty of \extension{dtx} files
	documented with \ltxclass{ltxguide}. However, I haven't yet used
	\docstrip, so my experience here is limited.

	\item[ltxguide] Provides several ergonomic features absent
	in \ltxclass{ltxdoc}. However, \ltxclass{ltxguide} is almost entirely
	undocumented, a fact which is partially mitigated by the fact that
	it's only about 150 lines long. \ltx\ is, as the name implies, based
	on \ltxclass{ltxguide}.
\end{classes}
And supporting packages include:
\begin{packages}
	\item[hypdoc] One of many, many packages by Heiko Oberdiek.
	\ctan{hypdoc} undertakes the ambitious task of patching the
	\ctan{doc} package in order to generate better indexes. In my
	experience, \ctan{hypdoc} is not compatible with
	\ltxclass{ltxguide}; as such, it isn't loaded in \ltx.

	\item[doctools] Provides many useful secondary commands such as
	|\ltxclass|, |\package|, and so on. Many are duplicated here.

	\item[showexpl] Provides the |LTXexample| environment which typesets
	\LaTeX\ code and displays it in a listing side-by-side.
	\ctan{showexpl} provides the functionality of \ctan{listings}'
	|\lstsample| command and more. \ctan{showexpl} does, however, rely
	on the fairly hefty \ctan{listings} package.
\end{packages}

\section{The \ltx\ document class}
Although \ltx\ provides many useful commands, much of its utility is in its
aesthetics. Much \LaTeX\ documentation is very ugly because producing
beautiful documentation requires significantly more code than most package
authors are interested in writing. This document is written with \ltx\ and
one package loaded (the \ctan{bera} font family).

\section{New commands}
\ltx\ provides several new commands for convenience.

\begin{desc}
|\email{<email>}|
\end{desc}

Typesets an email address with a |mailto:| link.

\begin{example}
	Emails, along with other hyperlinks, are colored |magenta|, although
	\ltx's default magenta is a bit closer to purple.
\begin{LTXexample}
\email{rebeccaturner@brandeis.edu}
\end{LTXexample}
\end{example}

\begin{desc}
|\https{<url>}|\qquad|\http{<url>}|
\end{desc}

Typesets |<url>| with |https://| or |http://| prepended to the link address;
this makes links display a bit prettier than |\url| might.

\begin{example} The following two listings are equivalent:
\begin{LTXexample}
\https{ctan.org}
\end{LTXexample}
\begin{LTXexample}
\href{https://ctan.org}{ctan.org}
\end{LTXexample}
\end{example}

\begin{desc}
|\ctan{<package>}|
\end{desc}

Typesets a package name with a link to |ctan.org/pkg/<package>|.

\begin{warning}
	\ltx's definition of |\ctan| differs from \ltxclass{ltxguide}'s,
	which simply typesets ``\textsc{ctan}'' in small-caps.
\end{warning}

\begin{LTXexample}
the \ctan{listings} package\dots
\end{LTXexample}

\begin{desc}
|\package{<package>}|\\
|\ltxclass{<document class>}|\\
|\option{<option name>}|\\
|\filename{<filename>}|\\
|\extension{<file extension>}|
\end{desc}

Typesets a \LaTeX\ package, option, file extension, etc.\ in |\texttt|.

\begin{note}
	Unlike those defined in the \ctan{doctools} package, these commands
	don't add entries to the index.
\end{note}

\begin{LTXexample}
\extension{tex} files
\end{LTXexample}

\begin{desc}
|\begin{warning}...\end{warning}|\\
|\begin{note}...\end{note}|\\
|\begin{example}...\end{example}|\\
|\begin{bug}...\end{bug}|
\end{desc}

Typeset ``notices'' with a hanging indent. Original definitions written by
Javier Bezos for the \ctan{enumitem} documenation. |\ltxguidex@noticestyle|
is executed before the marker text (``warning,'' ``note,'' etc.) in a
separate group. New notice environments can be created with |\newnotice|.

\begin{bug}
	If the first content in a notice environment is vertical, the marker
	text is hidden. This can be avoided by starting the
	environment with |\leavevmode\\| or by adding some introductory
	material to the first line.
\end{bug}

\begin{example}
	Although this example is short, note that subsequent lines will
	be indented. These environments only vary by text.

\begin{LTXexample}
\begin{warning}
    Lorem ipsum\dots
\end{warning}
\end{LTXexample}
\end{example}

\begin{desc}
|\newnotice{<environment name>}{<marker text>}|
\end{desc}

Creates a new notice environment in the style of |warning|, |note|, and so
on. The marker text is automatically uppercased.

\begin{desc}
|\begin{desc}...\end{desc}|
\end{desc}

Describes a command or environment, setting it out into the margin.

\end{document}
