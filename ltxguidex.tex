\documentclass{ltxguidex}
\usepackage{bera}
\newcommand{\ltx}{\ltxclass{ltxguidex}}
\newcommand{\ltxguide}{\ltxclass{ltxguide}}
\title{Towards Better \LaTeX\ Documentation With the \ltx\ Document Class}
\author{Rebecca Turner\thanks{Brandeis University;
	\email{rebeccaturner@brandeis.edu}}}
\date{2019-01-07}
\begin{document}
\maketitle

\begin{abstract}
	The \ltx\ document class provides\dots

	This document is written with the \ltx\ document class.
\end{abstract}

\begin{note}
	This release of \ltx\ is an experimental public beta; it intends to
	demonstrate a hopeful new direction without committing to a stable
	public \textsc{api}.

	Although \ltx\ is now suitable for use in your own documentation, do
	not be surprised if future versions break your docs.
\end{note}

\section{The State of the Docs}

\LaTeX\ documentation is easy enough to write that --- in general --- nobody
has bothered to package the ways of making it better. If one examines the
documentation for their favorite package, they'll likely find a few command
definitions that make some aspect of documentation writing more ergonomic.

\LaTeX\ documentation is enabled with two document classes and several
packages. Document classes include:
\begin{classes}
	\item[ltxdoc] Defines very little other than a few
	shorthands for documenting commands. Designed to be integrated with
	the \docstrip\ system, but I've seen plenty of \extension{dtx} files
	documented with \ltxguide. However, I haven't yet used
	\docstrip, so my experience here is limited.

	\item[ltxguide] Provides several ergonomic features absent
	in \ltxclass{ltxdoc}. However, \ltxguide\ is almost entirely
	undocumented, a fact which is partially mitigated by the fact that
	it's only about 150 lines long. \ltx\ is, as the name implies, based
	on \ltxguide.
\end{classes}
And supporting packages include:
\begin{packages}
	\item[hypdoc] One of many, many packages by Heiko Oberdiek.
	\ctan{hypdoc} undertakes the ambitious task of patching the
	\ctan{doc} package in order to generate better indexes. In my
	experience, \ctan{hypdoc} is not compatible with
	\ltxguide; as such, it isn't loaded in \ltx.

	\item[doctools] Provides many useful secondary commands such as
	|\ltxclass|, |\package|, and so on. Many are duplicated here.

	\item[showexpl] Provides the |LTXexample| environment which typesets
	\LaTeX\ code and displays it in a listing side-by-side.
	\ctan{showexpl} provides the functionality of \ctan{listings}'
	|\lstsample| command and more. \ctan{showexpl} does, however, rely
	on the fairly hefty \ctan{listings} package.

	Compare to more ``plain'' \LaTeX\ documentation, \ltx\ documentation
	can be expected to compile somewhat slower. This author is of the
	opinion that the improvements are so numerous that the slow-down is
	worth it.
\end{packages}

\section{The \ltx\ document class}
Although \ltx\ provides many useful commands, much of its utility is in its
aesthetics. Much \LaTeX\ documentation is very ugly because producing
beautiful documentation requires significantly more code than most package
authors are interested in writing. This document is written with \ltx\ and
one package loaded (the \ctan{bera} font family). Because \ltx\ is written
with inherent beauty, it ends up being a bit heavier than its competitors;
notably, it loads \ctan{xcolor}, \ctan{listings}, \ctan{graphicx}, and
\ctan{calc} by default.

\section{Commands provided by \ltxguide}

\begin{note}
	In \ltxguide, pipe characters (\texttt{\pipe}) mark verbatim text.

	However, between two pipes, the angle brackets |<<| and |>>| typeset
	as pretty angle brackets with regular italics between them;
	therefore, \texttt{\pipe}|<<package>>|\texttt{\pipe} typesets as
	|<package>|.

	To write literal angle brackets, simply double the characters;
	\texttt{\pipe}|<<<<|\texttt{\pipe} will render as |<<|  and
	\texttt{\pipe}|>>>>|\texttt{\pipe} will render as |>>|.

	To write literal pipe characters, use the |\pipe| command.
	\ltxguide\ uses \ctan{shortvrb} to supply this verbatim code. There
	are some small conflicts with \ltx's use of the \ctan{listings}
	package (in particular, pipes are silently gobbled from
	|lstlistings| environments, although they work normally within
	|verbatim|), which will hopefully be resolved with a coming change
	to \package{listings}; this simply depends on how quickly Jobst
	Hoffmann emails me back.
\end{note}

\section{New commands}
\ltx\ provides several new commands for convenience.

\begin{desc}
|\begin{desc}...\end{desc}|
\end{desc}

Describes a command or environment, setting it out into the margin and
surrounding it with a frame. Originally written by Javier Bezos for the
\ctan{enumitem} documenation.

\begin{example}
	Unfortunately, a side-by-side listing doesn't seem to be possible
	here because pipes seem to be gobbled by the \package{listings}
	package (a side-effect of loading both \ctan{listings} and
	\ctan{shortvrb}, perhaps). However, here's how the |\email| command
	is described in this document:

\begin{verbatim}
\begin{desc}
|\email{<<email>>}|
\end{desc}
\end{verbatim}
\end{example}

\begin{desc}
|\email{<email>}|
\end{desc}

Typesets an email address with a |mailto:| link.

\begin{example}
	Emails, along with other hyperlinks, are colored |magenta|, although
	\ltx's default magenta is a bit closer to purple.
\begin{LTXexample}
\email{rebeccaturner@brandeis.edu}
\end{LTXexample}
\end{example}

\begin{desc}
|\https{<url>}|\qquad|\http{<url>}|
\end{desc}

Typesets |<url>| with |https://| or |http://| prepended to the link address;
this makes links display a bit prettier than |\url| might.

\begin{example} The following two listings are equivalent:
\begin{LTXexample}
\https{ctan.org}
\end{LTXexample}
\begin{LTXexample}
\href{https://ctan.org}{ctan.org}
\end{LTXexample}
\end{example}

\begin{desc}
|\ctan{<package>}|
\end{desc}

Typesets a package name with a link to |ctan.org/pkg/<package>|.

\begin{warning}
	\ltx's definition of |\ctan| differs from \ltxguide's,
	which simply typesets ``\textsc{ctan}'' in small-caps.
\end{warning}

\begin{LTXexample}
the \ctan{listings} package\dots
\end{LTXexample}

\begin{desc}
|\package{<package>}|\\
|\ltxclass{<document class>}|\\
|\option{<option name>}|\\
|\filename{<filename>}|\\
|\extension{<file extension>}|
\end{desc}

Typesets a \LaTeX\ package, option, file extension, etc.\ in |\texttt|.

\begin{note}
	Unlike those defined in the \ctan{doctools} package, these commands
	don't add entries to the index.
\end{note}

\begin{LTXexample}
\extension{tex} files
\end{LTXexample}

\begin{desc}
|\begin{warning}...\end{warning}|\\
|\begin{note}...\end{note}|\\
|\begin{example}...\end{example}|\\
|\begin{bug}...\end{bug}|
\end{desc}

These environments typeset ``notices'' with a hanging indent. Original
definitions written by Javier Bezos for the \ctan{enumitem} documenation.
|\ltxguidex@noticestyle| is executed before the marker text (``warning,''
``note,'' etc.) in a separate group. New notice environments can be created
with |\newnotice|.

\begin{bug}
	If the first content in a notice environment is vertical, the marker
	text is hidden. This can be avoided by starting the
	environment with |\leavevmode\\| or by adding some introductory
	material to the first line.

	This is actually a bug in the |\list| command that the notice
	environments use.
\end{bug}

\begin{example}
	Although this example is short, note that subsequent lines will
	be indented. These environments only vary by text.

\begin{LTXexample}
\begin{warning}
    Lorem ipsum\dots
\end{warning}
\end{LTXexample}
\end{example}

\begin{desc}
|\newnotice{<environment name>}{<marker text>}|
\end{desc}

Creates a new notice environment in the style of |warning|, |note|, and so
on. The marker text is automatically uppercased.

\begin{desc}
|\begin{LTXexample}[<options>]...\end{LTXexample}|
\end{desc}

Typesets \LaTeX\ code next to a listing of its source. Providing examples
makes your user's lives easier, and should be done as much as possible. The
|LTXexample| environment is provided by the \ctan{showexpl} package.
Excerpted from \ctan{showexpl}'s documentation as of v0.3o 2016/12/11, valid
options include:

\begin{options}
	\item[attachfile] Boolean valued key, default value: false. If set to
	true the sourcecode will be attached to the \extension{pdf}
	file---presumed that the document is processed by |pdflatex|.
	\item[codefile] Name of the (temporary) file that contains the code
	which will be formatted as source code. The default value is |\jobname.tmp|.
	\item[explpreset] A |<key val list>| which serves for presetting the
	properties of the formatting of the source code, for values see the
	documentation of the \ctan{listings} package. The default value is
	empty.\footnote{\ltx\ redefines the default to perform syntax
	highlighting for \LaTeX, in addition to the general improvements
	made for all listings in the document.}
	\item[graphic] Name of a (graphic) file. This file---if present---will
	be included and displayed instead of the formatted code. The default value is empty.
	\item[hsep] Defines the horizontal distance between the source code and the
	formatted text.
	\item[justification] Defines the justification of the formatted text:
	reasonable values are |\raggedleft|, |\raggedright|, |\centering|. The
	default value is |\raggedright|.
	\item[overhang] A \textit{dimen}-value that defines the amount by which
	the formatted text and the source code can overlap the print space. The
	default value is 0\,pt.
	\item[pos:] Defines the relative position of the formatted text
	relating to the source code. Allowed values are |t|, |b|, |l|, |r|,
	|o|, and |i| for top, bottom, left, right, outer, and inner. The last
	values give sense only for two-sided printing, where there are outer
	and inner margins of a page. The default value is |l|.
	\item[preset] Any \TeX\ code executed before the sample code but
	not visible in the listings area.
	\item[rangeaccept] Boolean valued key, default value is false. If set
	to  true, one can define ranges of lines that will be excerpted from
	the source code.
	\item[rframe] Defines the form of the frame around the formatted
	text. With a non-empty value (e.\,g. ``single'') a simple frame
	will be drawn. In the future more kinds of frames will be supported.
	The default value is empty (no frame).
	\item[varwidth] Boolean valued key, default value is false. If set to
	true, the formatted text is set with its ``natural'' width instead of a
	fixed width as given by the value of the option |width|.
	\item[vsep] Defines the vertical distance between the source code and the
	formatted text.
	\item[wide] Boolean valued key, default value is false. If set to
	true, the source code and the formatted text overlap the print space
	and the margin area.
	\item[width] A \textit{dimen} value that defines the width of the
	formatted text. The default value depends of the relative positions of
	the source code and the formatted text.
	\item[scaled] Without a value the formatted text will be scaled to fit
	the given width of the result area. With a number as value the formatted
	text will be scaled by this number.
\end{options}

In addition to these options the kind of the result box (default: |\fbox|)
can be changed. For example:
\begin{latexcode}
\renewcommand\ResultBox{\fcolorbox{green}{lightgray}}
\setlength\ResultBoxSep{5mm}%  default: \fboxsep
\setlength\ResultBoxRule{2mm}% default: \fboxrule
\end{latexcode}

\begin{desc}
|\begin{packages}...\end{packages}|\\
|\begin{classes}...\end{classes}|\\
|\begin{options}...\end{options}|\\
\end{desc}

Frequently, package authors need to describe a series of options, packages,
or document classes. These environments wrap the |description| environment
and provide an |\item| which wraps a command like |\package|. In the
|packages| environment, |\item[listings]| translates to
|\item[\package{listings}]|.

\begin{LTXexample}
\begin{options}
    \item[foo] \dots
    \item[bar] \dots
\end{options}
\end{LTXexample}

\begin{desc}
|\begin{advise}...\end{advise}| $\equiv$\\
|\begin{advice}...\end{advice}| $\equiv$\\
|\begin{faq}...\end{faq}|\\
\end{desc}

Roughly copied from \ctan{listings}' internal \package{lstdoc} package,
these environments represent a list of questions and answers.

\begin{LTXexample}
\begin{faq}
\Q Lorem ipsum dolor sit amet?
\A Consectetur adipiscing elit, sed do eiusmod tempor incididunt ut labore et dolore magna aliqua.

\Q Ut enim ad minim veniam, quis nostrud?
\A Exercitation ullamco laboris nisi ut aliquip ex ea commodo consequat.
\end{faq}
\end{LTXexample}

Within these environments, |\Q| and |\A| indicate a question and an answer;
they're defined to |\item| and |\advisespace|, respectively.

\begin{note}
	|advice| and |faq| are exact synonyms for |advise|; in addition, the
	command |\advicespace| has the same meaning as the more British
	|\advisespace| within the environment.

	The list label for the |advise| environment is |\labeladvise|.

	The font is set with |\advisestyle|.
\end{note}

\end{document}
